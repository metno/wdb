\documentclass[a4paper,11pt]{wdb}

%----------------------------------------------------------------------------

\begin{document}

%----------------------------------------------------------------------------

\system{wdbAdmin}
\title{System Design Specification}
\author{Vegard B�nes}

\maketitle

%----------------------------------------------------------------------------

\maketoc

%----------------------------------------------------------------------------

\section{Introduction}

This document is the System Design Specification of the various WDB
Administartion programs.

\subsection{Purpose}

The system design specification provides a comprehensive overview of
the design of the software system. This is used to ensure agreement
between the developers of the WDB system. It is also intended to
make it easier for new developers of WDB to become productive
developing on the system.

\subsection{Definitions and Terminology}

None.



%----------------------------------------------------------------------------

\section{System Overview}



%----------------------------------------------------------------------------

\section{Technical Platform}

The administration software requires the following operating environment:

\begin{itemize}
\item Hardware: Desktop PC
\item Operating System: Linux
\item PostgreSQL client v8.1
\item libpqxx
\item Boost
\end{itemize}



%----------------------------------------------------------------------------

\section{System Architecture}

The administration module is made so that several administration
programs may be used. Therefore, presentation is sharply separated
from business logic and database access.

The following section describes the architecture of the administration systems.

\subsection{User interface}

UI is separated from the rest of the program, and only provides a way
to interpret a user's desires, and present the result of an operation
back to the user. No business logic is placed here.

There may exist several user interfaces, but they all use the same set
of functions. These functions are described below.

\subsubsection{Command line}

This is a user interface, where the user may type commands into a
terminal, and get results back to stdout.

\subsection{Core}

All operations to be performed by an administration program should use
the same set of functions. These are here described as the
administation core. 


\subsubsection{Business logic}

The business logic uses resources, such as predefined
transactions or the grib loader, to perform varoius operations, that
are required by a user interface.


\subsubsection{Database access}

The administration core contains a number of predefined transactions
for getting various informtaion from the database. These are not ment
to be used by other components of the system than the administration
business logic.


\subsection{Program Interface}

The various UI frontends to the WDB administration program have
different ways of presenting their available operations, but they have
all operations in common. These operations are:

\begin{itemize}
\item [List keys]
  Get a list of all keys for the given reference time.
\item [List fields]
  Get the number of fields in database, keyed on reference time.
\item [Display loadable files]
  Show a list of loadable files, under a given directory.
\item [Load data]
  Load the given files or directories into wdb
\end{itemize}

In addition, varying criteria for what is a loadable file may be set.
Parameters for database connection may be set, like in any other
application connecting to WDB.

\subsubsection{Command line}

After having started the program, the user may type commands into the
shell. These are 

\begin{itemize}
\item keys - to list keys
\item list - t olist fields in database
\item load - to load data into database
\item show\_ loadable to display loadable files
\item validator to see or change which file validator is used.
\end{itemize}


%----------------------------------------------------------------------------

\section{Core Design}

The following section describes the core design of the WDB
administration system.

\subsection{User interface}

\subsubsection{Command line}

The command line based user interface is divided into two parts: Input
and output.

The input part, the CommandLineInterpretor, interprets commands
entered onto the command line. The interpretation is done in an
infinite loop, only terminating when the user has typed ``exit'' into
the command line.

Once a complete line has been entered by the user, the first word
on the command line is taken to be a command. This command is checked
against a command - function object map. In case of a match, the found
function object is invoked, and otherwise the outputter object is
requested to display an appropriate error message.

The function objects for performing the various actions are all
subclasses of the abstract class CommandLineInterpretor::Command,
which offers two methods: operator() and help(). The former delegates
performing of the intended command to the outputter. The latter
provides a short help string specifying the command's function and
syntax.

The outputting class, AdminCmdLineOutput, is both responsible for
calling the core's functions, and for presenting the result to the
user. The class has one method for each provided command line command,
which will typically send a requst to the core, and thereafter
displaying the result to the user, using stdout.

Also, the outputting class has a method, info(), which may be used to
display other messages. This method should be used instead of writing
directly to stdout.

\subsection{Internal functions}

\subsubsection{Administration operations}

The class AdminOperations provides all functions which should be
available from any administration program.

The functions mostly do one of three things: Wrap a call to the
database, check the file system for loadable files, or call an
external program, such as gribLoad.

The calls to the database are specified separately, in their own
files. They are run in their own transaction.



%----------------------------------------------------------------------------
% Dummy Insert for Possible Appendix

%\appendix
%\include{app1}
%\include{app2}

%----------------------------------------------------------------------------

\end{document}


